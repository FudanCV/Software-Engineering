\chapter{引言}
\section{系统概述}
随着互联网的高速发展与普及,人们的消费观念消费方式也随之改变,网上购物在用户消费中所占的比例正逐渐攀升。在书籍售卖这一行业,网上书城以其便捷的购物模式以及相对低廉的价格吸引了大批的用户,有着很好的商业前景,而本系统正是在这样的背景下所设计的。

电子商务是指在互联网、企业内部网和增值网上以电子交易方式进行交易活 动和相关服务活动,是传统商业活动各环节的电子化、网络化。电子商务包括电子货币交换、供应链管理、电子交易市场、网络营销、在线事务处理、电子数据交换、存货管理和自动数据收集系统。在此过程中,利用到的信息技术包括:互联网、外联网、电子邮件、数据库、电子目录和移动电话。

网上书城是电子商务系统的典型代表。由于图书种类繁多、需求分散、易于保管,是长尾理论应用于零售的典型案 网上书城是电子商务发展最早的成功应用之一,充分利用电子商务的优势 , 在买家、供应商、配送、银行电子支付系统间架起一座桥梁。

本网络购书系统为除了用户提供便捷的检索、下单、支付、退货等常见功能外,还包含了与银行、仓库、配送员之间的数据交换,实现所有业务逻辑的一体化,方便了企业对系统、对库存、财务的全方位管理。
\section{文档概述}
本项目为《软件工程》课程实践项目第一部分。小组成员用面向数据流的结 构化设计的方法为网上书城系统做出了需求分析、数据流描述和结构化设计。本文档对以上内容进行了阐述,是网上书城系统的进一步实现的基础。本文档详细描述了网络购书系统的需求规约,为下一步设计和具体编码奠定了基础。
\section{参考文献}
\texttt{钱乐秋,赵文耘,牛军钰}.  软件工程.  	\textit{清华大学出版社},2007.