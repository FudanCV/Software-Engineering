\chapter{项目需求}
\section{功能需求}
\subsection{书城用户需求}
\begin{itemize}
	\item 通过本系统的 Web/Mobile 链接访问系统,浏览查看书城的书籍信息
	\item 注册成为书城用户,点击购买书籍,享有退货权利
	\item 付费升级为书城高级用户,享受购买书籍时折扣优惠和其他书城的增值服务
\end{itemize}
\subsection{银行需求}
\begin{itemize}
	\item 能收到从书城系统发出的付费请求,允许请求中的用户在银行的支付平台上完成一系列支付流程
	\item 能和书城进行实时对账提供及时的支付状态通知
\end{itemize}
\subsection{仓库管理需求}
\begin{itemize}
	\item 对书城各地书籍进行调度管理,包括出货、进货,生成发货单、进货单,尽可能保证不同城市间不同人群的需求得到最大可能的满足
	\item 对于用户完成的订单合理地安排出货和配送,用最短的时间和最低的成本完成书籍的配送流程
\end{itemize}
\subsection{配送员需求}
\begin{itemize}
	\item 能获取到最新的订单信息和仓库方生成的配送单,配送到用户,更新订单状态
	\item 获取用户的退货书籍,发回到最近的仓库,更新订单状态
\end{itemize}
\subsection{推荐算法需求}
\begin{itemize}
	\item 获取所有用户的浏览记录,推荐最受用户喜欢查看的图书
	\item 获取所有用户的购买记录,推荐最受用户喜欢购买的图书
	\item 获取当前的浏览记录和购买记录,利用协同过滤算法计算出用户可能喜欢的其他图书,或和用户具有高相似度的其他用户的推荐图书
\end{itemize}
\section{性能需求}
\subsection{同步性需求}
\begin{itemize}
	\item 用户的购买行为以及预定行为受限于当前库存,以避免仓库发不出货的系统错误。故对图书库存量的修改必须为原子操作,且不允许进行并发修改
	\item 系统多处涉及资金流动,需要不断请求核实,保证财务信息的一致性
	\item 仓库、物流配送与浏览模块之间的信息需要保持同步,此行为涉及到多个信息终端,	通过公用数据库表实现
\end{itemize}
\subsection{实时性需求}
\begin{itemize}
	\item 库存、预定、订单、进发货单、配送单之间的数据交互必须是实时的
	\item 当库存更新时,就需查询是否有用户预定的书目到库,若有则进行到货通知
	\item 当有图书形成订单时,要从库存中扣除对应数量,以避免多个用户同时对同一本书下单,导致发货失败
\end{itemize}
\subsection{灵活性需求}
系统需要接受的信息量很大,来源广,而且系统接收信息的来源可能随着书城的运营而不断扩展,可能需要随时对以下变化进行修改:
\begin{itemize}
	\item 图书的类型增减(修改多级导航)
	\item 各仓库图书数量变动(改变图书调度的策略)
	\item 新的支付方式(如更多的类支付宝产品)
	\item 新的客户端操作系统
	\item 网站首页每日变动
	\item 广告信息变动
\end{itemize}
\subsection{安全性需求}
\begin{itemize}
	\item 合理收紧各种权限包括数据库、服务器、应用后台、SVN 等权限,只把权限开放给需要使用的人
	\item 设置防火墙和严格的身份验证系统,保障用户的隐私和账户安全,并对网站进行定期维护,确保系统的安全与可靠
	\item 完善的存储机制来妥善保管好所有的日志
	\item 应对可能存在的恶意攻击、恶意数据抓取
\end{itemize}
\subsection{负载均衡需求}
当网站进行促销活动,出现购书、浏览高峰时,系统需要良好的应对,通过技术手段分流服务器、数据库压力,避免崩溃
\subsection{可维护性需求}
系统的各个功能高度模块化,各模块功能相对独立,实现高内聚低耦合。系统基本实现自主运行与自动纠错,降低运行成本与维护成本。而且因为模块功能的独立性,保证了系统不会因为单个模块的错误而整体瘫痪,也方便了测试维护过程。维护人员可对各模块进行逐一维护,而不必调整整个系统。